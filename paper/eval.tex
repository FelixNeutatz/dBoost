% Evaluation

We implemented dBoost in approximately 2000 lines of python3 code.
Our code is publicly available on GitHub under version 3 of the GNU Public License~\cite{github}.
The program loads tuples from a text file (column separators can be specified on the command line, so many formats are possible), analyzes the data, and reports outliers on the command line.
Table~\ref{table:flags} shows the usage options to run the models supported by dBoost.

\begin{table*}
\label{table:flags}
\caption{dBoost command line usage.}
\centering
\begin{tabular} {| l | l | p{10cm} |}
\hline
\multicolumn{3}{|c|}{./dboost-stdin.py input-file} \\
\hline
Flag & Options & Explanation \\
\hline
--gaussian & n\_stdev & Report outliers that fall more than n\_stdev standard deviations away from the mean of the data \\
--mixture & n\_subpops & Use a model of n\_subpops gaussians \\
--histogram & peak\_s & Consider only fields with a peaked distribution with peakiness peak\_s \\
  & outlier\_s & Report values that fall in classes with less than outlier\_s percent \\
--statistical & epsilon & Give hints to the model for correlations with Pearson R coefficient greater than epsilon \\
\hline
\end{tabular}
\end{table*}

We ran our tool on two real-world data sets: a lab directory and a collection of sensor data. The lab directory represents a heterogeneous data set with a significant amount of rich string data; we used it to evaluate the use of the Histogram model. The sensor data was used to evaluate the effectiveness of the statistical analyzer and the Gaussian and Mixture models.
