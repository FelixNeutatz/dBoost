% THIS IS AN EXAMPLE DOCUMENT FOR VLDB 2012
% based on ACM SIGPROC-SP.TEX VERSION 2.7
% Modified by  Gerald Weber <gerald@cs.auckland.ac.nz>
% Removed the requirement to include *bbl file in here. (AhmetSacan, Sep2012)
% Fixed the equation on page 3 to prevent line overflow. (AhmetSacan, Sep2012)

\documentclass{vldb}
\usepackage{graphicx}
\usepackage{balance}  % for  \balance command ON LAST PAGE  (only there!)
\usepackage{url}


\begin{document}

% ****************** TITLE ****************************************

\title{Outlier Detection in Real Time}

% possible, but not really needed or used for PVLDB:
%\subtitle{[Extended Abstract]
%\titlenote{A full version of this paper is available as\textit{Author's Guide to Preparing ACM SIG Proceedings Using \LaTeX$2_\epsilon$\ and BibTeX} at \texttt{www.acm.org/eaddress.htm}}}

% ****************** AUTHORS **************************************

% You need the command \numberofauthors to handle the 'placement
% and alignment' of the authors beneath the title.
%
% For aesthetic reasons, we recommend 'three authors at a time'
% i.e. three 'name/affiliation blocks' be placed beneath the title.
%
% NOTE: You are NOT restricted in how many 'rows' of
% "name/affiliations" may appear. We just ask that you restrict
% the number of 'columns' to three.
%
% Because of the available 'opening page real-estate'
% we ask you to refrain from putting more than six authors
% (two rows with three columns) beneath the article title.
% More than six makes the first-page appear very cluttered indeed.
%
% Use the \alignauthor commands to handle the names
% and affiliations for an 'aesthetic maximum' of six authors.
% Add names, affiliations, addresses for
% the seventh etc. author(s) as the argument for the
% \additionalauthors command.
% These 'additional authors' will be output/set for you
% without further effort on your part as the last section in
% the body of your article BEFORE References or any Appendices.

\numberofauthors{3} 

\author{
% You can go ahead and credit any number of authors here,
% e.g. one 'row of three' or two rows (consisting of one row of three
% and a second row of one, two or three).
%
% The command \alignauthor (no curly braces needed) should
% precede each author name, affiliation/snail-mail address and
% e-mail address. Additionally, tag each line of
% affiliation/address with \affaddr, and tag the
% e-mail address with \email.
%
% 1st. author
\alignauthor
Rachael Harding\\
       %\affaddr{Massachusetts Institute of Technology}\\
       %\affaddr{}\\
       %\affaddr{}\\
       \email{rhardin@mit.edu}
% 2nd. author
\alignauthor
Zelda Mariet\\
       \email{mariet@mit.edu}
% 3rd. author
\alignauthor
Clement Pit--Claudel\\
       \email{cpitcla@mit.edu}
%\and  % use '\and' if you need 'another row' of author names
}
\date{10 December 2014}


\maketitle

\begin{abstract}
Databases store large amount of data generated by humans or sensors.
Unfortunately, these data sets are prone to input errors such as human error or faulty sensors.
These errors are \emph{outliers}, data points that exhibit surprising behavior compared to the rest of the data.
We propose a system that automatically detects outliers.
The system can be used to find outliers in an existing database or, after a brief analysis on a small subset of the data, determine whether new data is an outlier as it is added to the database.
We believe that detecting outliers at run time and allowing them to be corrected immediately eases the burden on the database administrator and later analysis of the data.
In this project, we build a tool to facilitate the automatic detection of outliers.
We evaluate our tool's effectiveness at detecting outliers on a variety of data sets.
Our implementation is publicly available under the GNU Public License.
\end{abstract}

\section{Introduction}
% INTRODUCTION

%Detecting outliers is an important problem in detecting inconsistencies within a data set.
Sensor measurement errors, abnormal stock market fluctuations, credit card fraud, and human input error all result in outliers that can be detrimental if gone undetected.
These outliers, data that exhibits unexpected or surprising behavior compared to other data, 
Formally, an outlier is defined as “an observation which deviates so much from the other observations as to arouse suspicions that it was generated by a different mechanism” \cite{Hawkins1980}.
While previous work has focused on detecting outliers before performing data analysis, no tools to our knowledge have been built to not only automatically detect, but explain why outliers fall outside the expected data behavior.
To our knowledge, no tools have been built to allow the user or database administrator to be allowed to adjust data inputs at run time.


In this work, we draw on some of the ideas from existing literature on outlier detection and statistical and probabilistic methods to detect outliers and find inconsistencies in the data.
We combine these into a framework that makes three passes over the data in order to analyze, model, and find outliers in the data.
The beauty of the framework is that it can naturally be expanded so that analysis and modeling can be done offline on a small sampling of the dataset.
The models can then be applied to new data that is added to the database, so outliers can be detected without re-running any analysis on the rest of the data.

Our contributions are as follows:
\begin{enumerate}
\item We present a framework that detects data outliers.
\item We apply machine learning techniques to model data behavior.
\item We evaluate our framework on several real-world datasets.
\end{enumerate}

We built a tool that utilizes our framework.
This tool is publicly available under the GNU Public License~\cite{github}.

\section{Related Work}
% RELATED WORK

% Outlier Detection
There has been substantial research in how to build models to detect outliers \cite{Aggarwal2013}, including how to detect outliers in high-dimensional data by searching the subspaces of the data \cite{Zhang2004}\cite{Kriegel2009}.
However many of these algorithms are complex and can require substantial computation to determine whether a new data point lies outside the data.

Several algorithms exist in the data mining community to determine outliers when doing data analysis.
Local outlier factor measures the degree to which a data point is an outlier~\cite{Breunig2000}.
Other techniques include k-nearest neighbor and cluster analysis.

% Outlier Detection Visualization
Research has been done to attempt to explain why outliers exist given properties of the original data~\cite{Wu}. Unlike our tool, Scorpion starts with user-defined outliers and works backwards to find potential explanations as to why the data points are outliers.

% Statistical methods
Statistical methods have been used to detect dependencies between columns of relational databases for the purpose of informing the query optimizer of potential data dependencies \cite{Ilyas2004}. These methods require only a small sample of the data to detect functional dependencies with high probability of correctness. The relatively low computation required by these algorithms makes them more amenable to detecting data anomalies in real time. However, these methods are better suited for numerical data~\cite{Hodge2004}

% Gaussian models
Gaussian Mixture Models have been used for outlier detection in multiple contexts~\cite{Lu2005,Roberts1994,Roberts1999}.

% String data
To the extent of our knowledge, the literature regarding outlier detection on non-numerical data is much less extensive. Some common approaches include identifying outliers using a similarity measure~\cite{Budalakoti2006}, Probabilistic Suffix Trees~\cite{Sun2006} and sequence alignment~\cite{Bouarfa2012}.

Some specialized work has focused on inferring domain-specific rules on highly specific data such as a sequence of UNIX commands~\cite{Lane1997a,Lane1997b}. By contrast, we take on a general-purpose approach that is capable of dealing with data as diverse as a set of names and office numbers to real-valued sensor data. Additionally, we analyze data without any additional information on its structure.

Overall, we differ from previous approaches in that we are capable of analyzing a very wide range of data and do not use predefined rules for outlier detection -- although adding user-defined rules is possible in our framework.


\section{Framework Overview}
\section{Overview of dBoost}
\label{sec:overview}

We designed a framework that analyzes, models, and detects outliers in data.
The whole system can be seen in Figure~\ref{fig:pipeline}.

\begin{figure*}
  \centering %TODO: Should this be full width?
  \paddedgraphics[width=.8\textwidth]{../graphics/pipeline.pdf}
  \caption{Framework pipeline}
  \label{fig:pipeline}
\end{figure*}

As a first step, tuples are read from the database and expanded: a set of type-dependent features is extracted for each field. These features express simple properties of the data, such as the length of a string, or the parity of an integer.

These expanded tuples are then analyzed in order to obtain simple statistical information, and to detect soft functional dependencies between different fields. The expanded tuples are then used to train one of three data models, with the help of the  statistics and correlation hints gathered at the previous stage.

Finally, the trained model is used to classify tuples into regular records and outliers; these tuples can be the ones the model was trained with, or future inputs to the database system.

From a high level view, our pipeline is implemented as a three-pass streaming algorithm, requiring no memory beyond that required to train the individual models.

The different stages of our pipeline are summarized as follows and described in detail in the following sections:

\begin{enumerate}
\item Preprocessing -- Tuples are expanded using knowledge about the database schema and field types (Section~\ref{sec:preprocessing}).
\item Statistical analysis -- The expanded data is analysed to gather basic statistics, along with correlation information. These statistics are used for modeling and outlier detection (Section~\ref{sec:statistical-analysis}).
\item Data modeling -- We apply various user-selected machine-learning algorithms to build models of the data (Section~\ref{sec:model-creation}).
\item Outlier detection -- Using the models built in the previous stage and user-provided sensitivity thresholds, we report outliers identified by the models trained during the previous stage (Section~\ref{sec:outlier-detection}).
\end{enumerate}

\section{Framework Implementation}
\subsection{Preprocessor}
% PREPROCESSOR

We preprocess the data by first discarding tuples that do not have the correct number of tuples.
We then use a tuple expander to gather more information from the data, which we describe in the next section.

\subsubsection{Tuple Expansion}

Often data stored in databases has meaning, such as dates or times.
However, the semantics of SQL are not expressive enough to use them in outlier analysis.
For example, the day of the week may be relevant information from a date in a banking application in which transactions are only completed on weekdays, but unless the database stores this information in a separate column, the information is lost.
We expand tuples in order to harness isolated information about the data that may not be easy to see when looking at the data in its original form.

In order to capture this extra information, we break down tuples into a series of sub-tuples based on the type of each column along with the original value, as shown in Figure~\ref{TODO}.
We determined these breakdowns based on our own intuition and the datasets we worked with.

Our tool currently expands three kinds of data types: strings, dates, and integers.
All other types are passed to the other stages in the framework without modification.
Our tool automatically determines the data type as tuples are read in, although the tool could be modified to get this information from table schemas.

The expansions we do on strings includes indicating case, determining whether there are digits in the string, and the string length.
Dates are broken down into year, month, day, hour, minute, seconds, and day of the week.
Integers are expanded to isolate individual bit activation and modularity.


\subsection{Statistical Analysis}
% STATISTICAL ANALYSIS
Following the tuple expansion phase, the new expanded tuples go through a statistical analysis phase. This phase collects two kinds of information: simple statistics on each column's values, and data on possible inter-column correlations.

The statistical information includes the average value of each column, its variance and its standard deviation. These statistics can then be used by the outlier models to improve their performance.

Furthermore, if there is more than one column in the original dataset, we run an additional analysis to determine whether there are correlations between columns in the expanded sub-tuples.

This analysis is done using the Pearson product-moment correlation. This simple statistical method is commonly used to find correlation between multiple values. It relies on the Pearson product-moment correlation coefficient, which measures linear correlation between two vectors.

Given two column vectors $X$ and $Y$, Pearson's coefficient $R$ is given by the following formula:

\begin{equation} 
\label{eqn:pearson}
R = \frac{\Covar(X,Y)}{\sqrt{\Var(X)\Var(Y)}}
\end{equation}

$R$'s value always lies between -1 and 1. An $R$ value close to 0 indicates little or no correlation. The closer the coefficient is to 1, the more closely correlated the columns are; the closer it is to -1, the more inversely correlated the columns are.

If a sufficient correlation or inverse correlation is found, the statistical analyzer sends information (``hints'') about the tuples and sub-tuples that are correlated to the outlier models.

Currently, we only detect correlations between expanded fields of different columns, as they are more likely to be meaningful. However, although some correlations in expanded fields of the same column are redundant (\texttt{isUppercase} and \texttt{isLowercase} for a string), some might be relevant (for example, a correlation between \texttt{weekday} and \texttt{month} for a date). We leave this exploration to future work.

In our implementation, all statistics and correlations are computed in memory using a single pass over the data; the expanded tuples are analyzed one row at a time, and the final statistics and correlation coefficients are computed after the last tuple has been processed.

The analyzer's results are accessible by any models used at later stages in the tool.

\subsection{Create Models}
% MODEL CREATION

\subsubsection{Histogram Model}
\subsubsection{Gaussian Model}
The Gaussian model assumes that each expanded column is independent from the others, and that all values in a column $c$ are independent values drawn from a normal distribution $\mathcal N(\mu_c, \sigma_c)$.

The model's parameters (a pair $(\mu, \sigma)$ for each numerical column) are obtained with one pass over the data, without loading the tuples in memory.

\subsubsection{Mixture Model}

The Mixture Model (MM) assigns a Gaussian Mixture Model (GMM) to each correlation between numerical valued-columns that has been detected during the statistical analysis. These GMMs model the probability distribution of the correlated values.

For example, if the preprocessor detects that field 2 of expanded column 1 and field 1 of expanded column 3 are correlated, the model will learn a GMM to model this correlation: the probability of obtaining $(X_1, X_2)$ as values for the two correlated fields is given by
\[\Pr(X_1, X_2) = \sum_{j=1}^{n} \pi_j \mathcal N(\mu_j, \Sigma_j)(X_1, X_2)\]
where $n$ is the number of components chosen for the GMM (we set $n=2$, as learning this parameter would severely increase the time necessary for learning the mixture model), and $\pi_j, \mu_j$ and $\Sigma_j$ are parameters of the GMM. 

The Mixture Model learns all  $\pi_j, \mu_j$ and $\Sigma_j$ parameters for each correlation detected during the preprocessing phase. This was implemented using python's \texttt{scikit-learn} library; the process loads all correlations to memory. We expect that on average, the set of correlations will be much smaller than the set of rows themselves, thus limiting memory usage.


\subsection{Outlier Detection}
% OUTLIER DETECTION
Models, once properly trained, are used for classification and detection of outliers -- either on incoming INSERT operations on a running system, or on existing rows (typically, the ones used during the model training phase). % LATER: Citation about databases sizes?

Given that databases can contain tables with several hundred columns, simply flagging a row as an outlier is insufficient: users cannot be expected to painstakingly analyze each outlying rows. Instead, the tool should automatically indicate which values in the row caused it to be flagged as an outlier.

\subsubsection{Simple Gaussian Modeling}
The simple Gaussian model measures how much each value differs from the mean computed in the preceding pass to flag outliers. Given a tolerance parameter $\epsilon$, a row is deemed an outlier if at least one of its attributes $a$ has a value $v_a$ such that 
\begin{align}
  |v_a - \mu_a| \ge \epsilon \cdot \sigma_a
  \label{eqn:gaussian-outlier}
\end{align}
where $\mu_a$ and $\sigma_a$ are the model's parameters for column $a$, as described in Section~\ref{sec:gaus_model}.

In this model, detecting which values are responsible for the outlier flag is simply a matter of keeping track of which attributes satisfied Equation~\ref{eqn:gaussian-outlier}.
 
\subsubsection{Mixture Modeling}
The Gaussian Mixture model uses the correlation hints provided by the statistical analysis phase to break down each row into small set of tuples of presumably correlated values.

Each resulting tuple is then evaluated using the corresponding GMM; if the correlated values' $v_c$ probability is smaller than the user-defined threshold parameter $\theta$; that is, if
\begin{align}
  \Pr(v_c | \pi_c, \mu_c, \Sigma_c) \leq \theta 
  \label{eqn:mixture-outlier}
\end{align}
then the row is flagged as an outlier.

As in the Gaussian Model, providing the user with a list of attributes that caused the row to be flagged as an outlier is simply a matter of tracking correlations that failed test~\ref{eqn:mixture-outlier}.

\subsubsection{Histogram Modeling}
The histogram-based modeling strategy proceeds in two phases to detect outliers. 

First, after running through the training phase, it decides for each histogram whether that histogram is peaked enough that it may be used to detect outliers. The aim of this phase is to discard histograms where most bins have a similar numbers of values. In practice, we use the following simple statistical test to determine whether a histogram is ``peaked'' (modal) enough: given a histogram with $N$ bins, we count the number $n_{top}$ of values in the top $N'$ bins, where $N'$ is $1$ for $1 \leq N \leq 3$, $2$ for $4 \leq N \leq 5$, and $3$ for $3 \leq N \leq 16$ (histograms with $N > 16$ bins were previously discarded). A histogram is deemed relevant to outlier detection if
\begin{equation}
 \frac{n_{top}}{n} \geq \theta 
\end{equation}
where $\theta$ is a user-chosen threshold. 

After identifying a relevant set of histograms (this operation only needs to run once, at the very beginning of the last pass), we proceed to the actual detection phase. We classify an expanded tuple $X$ as an outlier if any of its values (or set of values, as grouped according to the correlation hints previously obtained) $x_a$ verifies:
\begin{equation}
h_a(x_a) \le \epsilon \sum_k h_a(k)
\end{equation}
where $h_a(x)$ counts the number of tuples in bin $x_a$, and $\epsilon$ is a user-chosen sensitivity parameter.


\section{Evaluation}
\subsection{CSAIL Directory}
% CSAIL Directory
The CSAIL directory is an online directory of about 1000 faculty, staff and students in the MIT Computer Science and Artificial Intelligence Laboratory \cite{CSAILDirectory}. Each entry contains a person's name, phone number, office number, email address, and position.

Some data, such as a phone number, may be missing from the directory. Still, we expect our framework to be usefull in flagging discrepancies between the different records. Since the notion of what precisely constitutes an outlier here is imprecise at best, we also expect the tool to allow the user to explore different sets of parameters. To illustrate the process, we chose to dicuss the results returned by three runs of the tool, with increasingly strict limits on the number of outliers returned. Because the CSAIL test set is exclusively textual, we modeled it using histograms.

\subsubsection{Initial run: low specificity filtering}
The search for outliers is initiated with parameters $\theta = 0.8, \epsilon = 0.2$, by issuing the following command (the \lstinline{--statistical 1} parameter, used to disable correlation detection, is omitted for brevity in the following examples):

\begin{lstlisting}[gobble=2]
  ./dboost-stdin.py --histogram 0.8 0.2 
                    samples/csail.txt
\end{lstlisting}

This invocation produces a long list of outliers; a small subset of these is presented below; for privacy reasons, names have been omitted in the following listings. Phone numbers and emails have been removed and anonymized, respectively, and office numbers were exchanged.

\begin{lstlisting}[gobble=2]
  Doe, Jane, 32-D968, 
    $jdoe@CSAIL.MIT.EDU$, Postdoctoral Associate

  Sholmes, Herlock, $32-221B$, 
    shherlock@csail.mit.edu, Graduate Student

  $Evans-potter$, Lily, 32-G972, 
    lep@csail.mit.edu, Research Scientist
\end{lstlisting}

In total, the list covers 451 lines, out of a total of 1000. Office numbers are often flagged, as well as names, and email addresses. Before diving into the particulars of why these values are returned, we therefore reduce the result set by imposing more stringent thresholds for outlier detection. The new invocation of the tool is shown below: 

\begin{lstlisting}[gobble=2]
./dboost-stdin.py --histogram 0.9 0.05 
                  samples/csail.txt
\end{lstlisting}

This reduces the outliers set to $68$ results. Most of the hits against office numbers have disappeared: increasing the peak threshold to $0.9$ did not eliminate the corresponding histogram, but lowering the sensitivity did eliminate some of the questionable office numbers. At this point, it becomes interesting to look into the details of these outliers; adding the \lstinline{--verbose} flag gives us a bit more insight about these results:

\begin{lstlisting}[gobble=2]
  Sholmes, Herlock, $32-221B$, 
    shherlock@csail.mit.edu, Graduate Student
  > Value '32-221B' (3) doesn't match 
    feature 'signature'
 
  $Evans-potter$, Lily, 32-G972, 
    lep@csail.mit.edu, Research Scientist
  > Value 'Evans-potter' (1) doesn't match 
    feature 'title case'  
\end{lstlisting}

\lstinline{Doe, Jane} has disappeared from the list, since improperly cased e-mails are common enough in the database that they are not considered an outlier at sensitivity level $\epsilon = 0.05$.
 
Should further information be wanted, our tool offers an extra verbosity level, \lstinline{--debug};

\begin{lstlisting}[gobble=2]
  Sholmes, Herlock, $32-221B$, 
    shherlock@csail.mit.edu, Graduate Student
  > Value '32-221B' (3) doesn't match 
    feature 'signature'
  • histogram for ('signature',):
    ██████████ 
    Lu,Lu,Nd,Nd,Pd,Nd,Nd,Nd
    Lu,Nd,Nd,Pd,Nd,Nd,Nd
    Nd,Nd,Lu,Pd,Nd,Nd,Nd
    ████████████████████ Nd,Nd,Pd,Lu,Nd,Nd,Nd
    ██ Nd,Nd,Pd,Lu,Nd,Nd,Nd,Lu
    ██████ Nd,Nd,Pd,Nd,Nd,Nd
    $█ Nd,Nd,Pd,Nd,Nd,Nd,Lu$
    Nd,Nd,Pd,Nd,Nd,Nd,Nd
    Nd,Pd,Nd,Nd,Nd

  $Evans-potter$, Lily, 32-G972, 
    lep@csail.mit.edu, Research Scientist
  > Value 'Evans-potter' (1) doesn't match 
    feature 'title case'  
  • histogram for ('title case',):
    $False$
    ████████████████████ True
\end{lstlisting}

Our tool highlights the incorrect field, and prints the corresponding histogram. The bin in which the suspicious value falls is also highlighted. The \lstinline{signature} case in particularly interesting: to extract the signature of a string, our tool replace each character by the name of its Unicode class: uppercase Latin letters are \lstinline{Lu}, numbers are \lstinline{Nd}, and punctuation signs are \lstinline{Pd}; hence the string \lstinline{32-221B} is converted to \lstinline{Nd,Nd,Pd,Nd,Nd,Nd,Lu}, which does not fal in any of the dominant bins (the most frequent one, \lstinline{Nd,Nd,Pd,Lu,Nd,Nd,Nd}, describes office numbers of like \lstinline{32-G804}, the predominant form of office numbering in the Stata Center.  

Manual inspection of the results reveal that most of the outliers reported are actually values that require fixing. There are, however, a number of false positives, such as:

\begin{lstlisting}[gobble=2]
  $MacDonald$, Ronald, 32-D597, rmacdon@csail.mit.edu, Graduate Student    
  > Value 'MacDonald' (1) doesn't match 
    feature 'title case'  
  • histogram for ('title case',):
    $False$
    ████████████████████ True
\end{lstlisting}

The casing of \lstinline{MacDonald} is proper, but our tool notes that it doesn't adhere to the standard derived from almost all other tuples, and thus reports it. 


\subsection{Twitter}

\subsection{Intel Lab Data}
% INTEL LAB DATA

We evaluated our outlier detection framework on sensor data from the publicly available Intel Lab Data set \cite{IntelLabData}.
The Intel Lab Data contains data collected from 54 sensors spread throughout the Intel Berkeley Research Lab.
Each data entry is timestamped and contains information including humidity, temperature, light and voltage taken from a Micro2dot sensor and weatherboard.
The dataset includes approximately 2.3 million entries, but in our experiments we drew from a random sample of the rows.

This data set has known outliers from faulty sensor readings due to periods of critically low voltage.
Most notably, sensor 15 fails and 


\subsection{Presidential Campaign Finance}
% PRESIDENTIAL CAMPAIGN DATA
\cite{PresCampaignData}


\section{Future Work}
% FUTURE WORK

In our future work, we would like to expand our framework to include more statistical and probabilistic models.



\section{Conclusion}
% CONCLUSION

In this paper we presented dBoost, a toolkit that leverages tuple expansion to detect outliers in both numerical and heterogeneous data sets. We demonstrated that well-known artificial intelligence strategies could be used to flag spurious numerical and to a lesser extent non-numerical data. We demonstrated that simple correlation modeling was useful in inferring data dependencies and improving the accuracy of outlier detection procedures. We discussed histogram-based models, and showed that they provided a useful tool in analysing mostly textual data.

We showed that our toolkit performs well on real-world problems, including identifying potentially wrong entries in a people directory and flagging erroneous values generated by faulty sensors. Our toolkit and its source code are available for public use under a permissive license, with the hope of allowing database users to formulate their own type-based rules and find discrepancies in their own data.

We believe that the preliminary results presented in this paper are promising, especially in the area of identifying outliers in heterogeneous data.




% The following two commands are all you need in the
% initial runs of your .tex file to
% produce the bibliography for the citations in your paper.
\bibliographystyle{abbrv}
\bibliography{paper.bib}  % vldb_sample.bib is the name of the Bibliography in this case
% You must have a proper ".bib" file
%  and remember to run:
% latex bibtex latex latex
% to resolve all references

%\subsection{References}

%APPENDIX is optional.
% ****************** APPENDIX **************************************
% Example of an appendix; typically would start on a new page
%pagebreak

%\begin{appendix}

%\end{appendix}



\end{document}

