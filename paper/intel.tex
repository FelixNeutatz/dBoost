% INTEL LAB DATA

We evaluated our outlier detection framework on sensor data from the publicly available Intel Lab Data set \cite{IntelLabData}.
The Intel Lab Data contains data collected from 54 sensors spread throughout the Intel Berkeley Research Lab.
Each data entry is timestamped and contains information including humidity, temperature, light and voltage taken from a Micro2dot sensor and weatherboard.
%The dataset contains a total of approximately 2.3 million measurements.

The Intel lab dataset has known outliers from faulty sensor readings due to periods of critically low voltage.
Most notably, sensor 15 fails, its voltage drops, and the temperature rises to over 120 degrees Celsius while the humidity readings fail. 
For this experiment, we analyzed only the humidity, temperature, light and voltage readings from sensor 15.  
Our outlier detector captures the lowest voltage readings when running a Gaussian model with a tolerance of 3.
As we reduce the tolerance, we also capture the bad humidity readings and the the high temperature readings.

Another rogue sensor is sensor 18.
For this experiment, we used only the humidity, temperature, light and voltage readings from sensor 18 as our input dataset.
So much of the data is off in this data set that when we only analyze data from sensor 18, the model identifies the readings when the sensor is functioning correctly as outliers.

In a third experiment, we do statistical analysis and create models using a separate training data made of 1000 data points selected randomly from the approximately 2.3 million data points in the Intel lab data set.
When using the data from sensor 15 as the test set, the Gaussian model captures high light measurements as outliers at a tolerance around 2.5.
When the tolerance is reduced to 2, the Gaussian model outputs the bad humidity and high temperature readings as outliers.
When using the data from sensor 18 as the test set, the Gaussian model outputs the low voltage readings as outliers.

%Gaussian Mixture Model?
