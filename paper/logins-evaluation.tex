\subsubsection{Logins: a more realistic partitioned dataset}

Our web activity synthetic datasets are comprised of two columns: a \emph{Unix} timestamp, stored as an \texttt{INT}, and a country. Each dataset is supposed to track the connections of a registered user on a website; such a dataset could be obtained by selecting the relevant rows out of a large table listing all connections of all users. Each dataset exhibits a different pattern:

\begin{itemize}
\item One user always connects from the same country; values that do not match this country are outliers.
\item The second connects from one country during the week, and from another during the week-end; outliers in this case are connections from a country that doesn't match the country for that day of the week.
\item The third user connects from a set of three countries, with no discernible pattern. This should not return any outliers.
\end{itemize}

The datasets are randomly generated sets of 2000 most recent connections, listed in no particular order. The target outlier rate is \SI{5}{\percent} in each generated dataset.

Just like in the \emph{Fizz-Buzz} example, numerical models are useless here. In the first case, a histogram-based model with no correlation analysis is sufficient to flag the outliers. In the second case, the discrete statistical analysis phase singles out interesting pairs of correlated columns, including \texttt{(date\,\#\,day of week, country)} and \texttt{(date\,\#\,is weekend, country)}. A histogram-based model is sufficient to successfully flag outliers, without resorting to partitioning.

\fxnote{Exclude self-correlations from the discrete analyzer to speed it up and get better results}

Mixing two of these datasets, however, shows the limits of the non-partitioned histogram approach. If we only look at two-columns correlations the individual behavior patterns become less apparent, and if we look at three-column correlations the histograms either become too large, or spurious hits start to appear due to the many discrete correlations hints returned by the analyzer.

%TODO
