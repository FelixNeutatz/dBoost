% INTRODUCTION

%Detecting outliers is an important problem in detecting inconsistencies within a data set.
Sensor measurement errors, abnormal stock market fluctuations, credit card fraud, and human input error all result in outliers that can be detrimental if gone undetected.
These outliers, data that exhibits unexpected or surprising behavior compared to other data, 
Formally, an outlier is defined as “an observation which deviates so much from the other observations as to arouse suspicions that it was generated by a different mechanism” \cite{Hawkins1980}.
While previous work has focused on detecting outliers before performing data analysis, no tools to our knowledge have been built to not only automatically detect, but explain why outliers fall outside the expected data behavior.
To our knowledge, no tools have been built to allow the user or database administrator to be allowed to adjust data inputs at run time.


In this work, we draw on some of the ideas from existing literature on outlier detection and statistical and probabilistic methods to detect outliers and find inconsistencies in the data.
We combine these into a framework that makes three passes over the data in order to analyze, model, and find outliers in the data.
The beauty of the framework is that it can naturally be expanded so that analysis and modeling can be done offline on a small sampling of the dataset.
The models can then be applied to new data that is added to the database, so outliers can be detected without re-running any analysis on the rest of the data.

Our contributions are as follows:
\begin{enumerate}
\item We present a framework that detects data outliers.
\item We apply machine learning techniques to model data behavior.
\item We evaluate our framework on several real-world datasets.
\end{enumerate}

We built a tool that utilizes our framework.
This tool is publicly available under the GNU Public License~\cite{github}.
