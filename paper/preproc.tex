% PREPROCESSOR

Before being used to generate a model of the data, the tuples are cleaned up and expanded in order to extract relevant information. 

\subsubsection{Tuple Expansion}

Relevant information about tuples stored in databases is not necessarily readily available given the storage format. For example, the day of the week might be an important information for outlier detection (for example in a banking application where a specific transaction only takes place on Mondays), but not obvious when working with an integer representing a Unix time stamp.

In order to capture this additional information, we break down each field in a tuple into a sub-tuple of extracted features, which depend on the type of the original field (see Figure~\ref{fig:TODO})

Our tool currently expands three kinds of data types: strings, dates, and integers. All other types are passed to the other stages in the framework without modification.

Currently, our tool automatically detects field type as tuples are read in; it could also be modified to get this information directly from table schemas.

The expansions we do on strings includes indicating case, determining whether there are digits in the string, and the string length.
Dates are broken down into year, month, day, hour, minute, seconds, and day of the week.
Integers are expanded to isolate individual bit activation and modularity.

