% PREPROCESSOR

We preprocess the data by first discarding tuples that do not have the correct number of tuples.
We then use a tuple expander to gather more information from the data, which we describe in the next section.

\subsubsection{Tuple Expansion}

\noindent\begin{minipage}{0.8\linewidth}
  \itshape
  What if you need to store a date and time value with subsecond resolution? MySQL
  currently does not have an appropriate data type for this, but you can use your own
  storage format: you can use the BIGINT data type and store the value as a timestamp in
  microseconds, or you can use a DOUBLE and store the fractional part of the second after
  the decimal point.
\end{minipage}
\begin{flushright}
  \textit{High-Performance MySQL}, 3\textsuperscript{rd} edition (2012), p127
\end{flushright}

Often data stored in databases has meaning, such as dates or times.
However, the semantics of SQL are not expressive enough to use them in outlier analysis.
For example, the day of the week may be relevant information from a date in a banking application in which transactions are only completed on weekdays, but unless the database stores this information in a separate column, the information is lost.
We expand tuples in order to harness isolated information about the data that may not be easy to see when looking at the data in its original form.

In order to capture this extra information, we break down tuples into a series of sub-tuples based on the type of each column along with the original value, as shown in Figure~\ref{TODO}.
We determined these breakdowns based on our own intuition and the datasets we worked with.

Our tool currently expands three kinds of data types: strings, dates, and integers.
All other types are passed to the other stages in the framework without modification.
Our tool automatically determines the data type as tuples are read in, although the tool could be modified to get this information from table schemas.

The expansions we do on strings includes indicating case, determining whether there are digits in the string, and the string length.
Dates are broken down into year, month, day, hour, minute, seconds, and day of the week.
Integers are expanded to isolate individual bit activation and modularity.

\begin{figure}
  \newenvironment{stackedlines}{\renewcommand{\arraystretch}{1.2}\begin{array}[b]{@{}l@{\quad}l@{}}}{\end{array}}
  $\begin{stackedlines}
    \texttt{string:}\\
    \texttt{\parbox{\widthof{1418222134.325}}{"32-G414"}}
  \end{stackedlines} \longrightarrow
  \begin{cases}
    \text{length: } & \texttt{7}\\
    \text{pattern: } & \texttt{NNPLNNN}\\
    \text{uppercase: } & \texttt{True (1)}\\
    \text{lowercase: } & \texttt{False (0)}\\
    \text{\parbox{\widthof{base-10 residue:}}{title case:} } & \texttt{True (1)}
  \end{cases}$

  $\begin{stackedlines}
    \texttt{int:}\\
    \texttt{\parbox{\widthof{1418222134.325}}{1418222134}}
  \end{stackedlines} \longrightarrow
  \begin{cases}
    \text{date: } & \texttt{(2014,12,10)}\\
    \text{time: } & \texttt{(14,35)}\\
    \text{weekday: } & \texttt{Wed (2)}\\
    \text{day of year: } & \texttt{344}\\
    \text{binary: } & \texttt{0b10101\ldots010}\\
    \text{base-10 residue: } & \texttt{4}
  \end{cases}$

   $\begin{stackedlines}
    \texttt{float:}\\
    \texttt{1418222134.325}
  \end{stackedlines} \longrightarrow
  \begin{cases}
    \text{intpart: } & \texttt{1418222134}\\
    \text{decpart: } & \texttt{0.325}\\
    \text{millis: } & \texttt{325}\\
    \text{\parbox{\widthof{base-10 residue:}}{date, \ldots:} } & \texttt{\ldots}
  \end{cases}$

  \caption{Summary of tuple expansion rules currently implemented.}
  \label{fig:tuple-expansion}
\end{figure}
