% OUTLIER DETECTION
Once the selected model has learned all necessary parameters, it can be used for classification purposes - either on new rows for real-time outlier detection, or on rows already present in the database (including those used for learning the parameters).

Given that databases can contain tables with several hundred columns, simply flagging a row as an outlier is insufficient: the user cannot be expected to painstakingly analyze each outlier row. Instead, the tool should automatically indicate which values in the row caused it to be flagged as an outlier.

\subsubsection{Histogram Model}
\subsubsection{Gaussian Model}
The Gaussian Model uses each numerical value's distance to the mean as its criterion for outlier detection. 

Given a parameter $\texttt{tolerance}$, a row is flagged as an outlier if at least one of its attributes $a$ has a value $v_a$ such that 
\begin{align}
|v_a - \mu_a| \leq \texttt{tolerance} \cdot \sigma_a
\label{eqn:gaussian-outlier}
\end{align}
doesn't hold, where $(\mu_a, \sigma_a)$ are the model's parameters for column $a$, as described in Section~\ref{sec:gaus_model}.

In this model, detecting which values are responsible for the outlier flag is simply a matter of keeping track of which attributes didn't satisfy equation~\ref{eqn:gaussian-outlier}.
 
\subsubsection{Mixture Model}
The Mixture Model uses the correlation hints provided by the statistical analysis phase to break down a row into subsets of correlated values.

Each correlation subset is then evaluated using the corresponding GMM; if the correlated values' $v_c$ probability is smaller than the user-defined parameter \texttt{threshold}
\begin{align}
\Pr(v_c | \pi_c, \mu_c, \Sigma_c) \leq \texttt{threshold}, 
\label{eqn:mixture-outlier}
\end{align}
the row is flagged as an outlier.

As in the Gaussian Model, providing the user with which attributes caused the row to be flagged as an outlier can be done by tracking correlations that failed test~\ref{eqn:mixture-outlier}.
