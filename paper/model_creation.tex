% MODEL CREATION

\subsubsection{Histogram Model}
\subsubsection{Gaussian Model}
\label{sec:gaus_model}
The Gaussian Model assumes that each value in a column $c$ is  independent from the others, and drawn from a normal distribution $\mathcal N(\mu_c, \sigma_c)$.

The model's parameters (a pair $(\mu, \sigma)$ for each numerical column) are obtained by calculating each column's mean and standard deviation, as calculated by the statistical analyzer. %, using a single pass over the data, without loading the tuples in memory.

\subsubsection{Mixture Model}

The Mixture Model (MM) assigns a Gaussian Mixture Model (GMM) to each correlation between numerical valued-columns that was detected during the statistical analysis. These GMMs model the probability distribution of the correlated values.

For example, if the statistical analysis phase detects that two fields in two different expanded columns are correlated, the model will learn a GMM to model this correlation. The probability of obtaining $(X_1, X_2)$ as values for the two correlated fields is given by
\[\Pr(X_1, X_2) = \sum_{j=1}^{n} \pi_j \mathcal N(\mu_j, \Sigma_j)(X_1, X_2)\]
where $n$ is the number of components chosen for the GMM ($n$ is a user-defined parameter), and $\pi_j, \mu_j$ and $\Sigma_j$ are parameters of the GMM. 

The Mixture Model learns all  $\pi_j, \mu_j$ and $\Sigma_j$ parameters for each correlation found during the statistical analysis phase. This was implemented using python's \texttt{scikit-learn} library. The learning process loads all correlations to memory, but not the rows themselves. We expect that on average, the set of correlations will be much smaller than the set of rows themselves, thus limiting memory usage.

